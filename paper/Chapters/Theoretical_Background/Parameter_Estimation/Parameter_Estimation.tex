% !TEX encoding = UTF-8 Unicode
% !TEX spellcheck = en_US
% !TEX root = ../../../ICMA2020.tex

\subsection{Parameter Estimation}
\label{subsec:ParameterEstimation}

The base parameter form of the dynamics in (\ref{eq:model_regressor}) is used to estimate
the parameter vector $\boldsymbol{\theta} \in \mathbb{R}^{35}$.
%
Using a sufficient number of samples in the estimation process leads to an over-determined system of linear equations:

\begin{equation} \label{eq:ParameterEstimation_problem}
\begin{aligned}
    \boldsymbol{y} &= \boldsymbol{C} \boldsymbol{\theta} + \boldsymbol{\epsilon}, \\
    \begin{pmatrix}
        \boldsymbol{\tau}(t_1) \\
        \vdots \\
        \boldsymbol{\tau}(t_N) \\
    \end{pmatrix} &= 
    \begin{pmatrix}
        \boldsymbol{X}(t_1) \\
        \vdots \\
        \boldsymbol{X}(t_N) \\
    \end{pmatrix} \boldsymbol{\theta} +
    \begin{pmatrix}
        \boldsymbol{e}(t_1) \\
        \vdots \\
        \boldsymbol{e}(t_N) \\
    \end{pmatrix}.
\end{aligned}
\end{equation}

Here $\boldsymbol{y}$ contains the $N$ samples of the torque vector $\boldsymbol{\tau}$, $\boldsymbol{C}$ denotes the design matrix and $\boldsymbol{\epsilon}$ is the vector of errors.

Equation \eqref{eq:ParameterEstimation_problem} can be solved for $\boldsymbol{\theta}$ using a least squares estimator. The weighted least squares (WLS)  \cite{Gautier.2013,M.Gautier.1995,V.Bargsten.2013,Swevers.1997} is used to account for different levels of noise in the measurements: 
\begin{equation} \label{eq:ParameterEstimation_WLS}
    \begin{aligned}
	    \hat{\boldsymbol{\theta}} &= \arg \min_{{\boldsymbol{\theta}}} (\boldsymbol{y} - \boldsymbol{C} \boldsymbol{\theta})^\mathrm{T} \boldsymbol{W} (\boldsymbol{y} - \boldsymbol{C} \boldsymbol{\theta}) \\
	    &= (\boldsymbol{C}^\mathrm{T} \boldsymbol{W} \boldsymbol{C})^{-1} \boldsymbol{C}^\mathrm{T} \boldsymbol{W} \boldsymbol{y}.
	\end{aligned}
\end{equation}
Matrix $\boldsymbol{W}$ contains the weights of the individual samples, see also Sec.\,\ref{subsec:SignalProcessing}.
