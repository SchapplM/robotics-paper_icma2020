% !TEX encoding = UTF-8 Unicode
% !TEX spellcheck = en_US
% !TEX root = ../../../ICMA2020.tex

\subsection{Dynamics Model of Serial Robots}
This subsection describes the rigid body dynamics model of the 6-axis robot.
The joint torque $\boldsymbol{\tau}$ of all axes required for the feedforward control can be calculated by
%
\begin{equation}
\label{eq:model_equation1}
\begin{split}
\boldsymbol{\tau}=
\boldsymbol{M}(\boldsymbol{q}) \ddot{\boldsymbol{q}}+\boldsymbol{c}(\boldsymbol{q}, \dot{\boldsymbol{q}})+\boldsymbol{g}(\boldsymbol{q})+\boldsymbol{h}(\dot{\boldsymbol{q}}).
\end{split}
\end{equation}
%
Herein $\boldsymbol{q}$, $\dot{\boldsymbol{q}}$ and $\ddot{\boldsymbol{q}}$ represent the joint angles, velocities and accelerations given by the motion planning. $\boldsymbol{M}$ contains the moments of inertia, $\boldsymbol{c}$ Coriolis effects and $\boldsymbol{g}$ gravitational effects. The friction model is represented by $\boldsymbol{h}$ with
%
\begin{equation}
h_{j}=f_{\mathrm{c}, j} \operatorname{sgn}\left(\dot{{q}}_{j}\right)+f_{\mathrm{v}, j} \dot{q}_{j},
\end{equation}
%
where $f_{\mathrm{c}, j}$ represents the Coulomb friction coefficient for joint $j$ and $f_{\mathrm{v}, j}$ the viscous friction coefficient.
The gear transmission between motor and link side is omitted for the sake of simplifying the equations. Measured motor velocities and estimated motor torques are transformed accordingly.

A minimal parametric model is used for the inverse dynamics  model. 
It has been obtained by symbolically determining the base parameters using the software \textsc{Maple}, similar to the geometrical approach of \cite{Khalil.2006}.
Thereby it is possible to identify the model parameters within one identification cycle.
Since some of the robot links possess symmetries, assumptions about their center of mass and products of inertia were made. From originally 60 parameters of the rigid body dynamics, these assumptions and the base parameter formulation led to only 23 base inertial parameters.
For each of the six joints there are two additional parameters representing the Coulomb and viscous friction. The chosen parameter vector $\boldsymbol{\theta}$ to be identified therefore contains 35 entries and is shown in appendix \ref{sec:MinparamVector}.
